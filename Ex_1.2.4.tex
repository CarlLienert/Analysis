\documentclass{article}
\usepackage[utf8]{inputenc}
\usepackage{amssymb}
\usepackage{amsmath}

\title{Exercise 1.2.4}
\author{Carl}
\date{April 2020}

\begin{document}

\maketitle

\section{Exercise 1.2.4}
Define $A_p$ for $p$ prime as follows:

\begin{align*}
    A_2&=2\mathbb{N}\\
    A_3&=3\mathbb{N}-A_2\\
    A_5&=5\mathbb{N}-(A_2 \cup A_3)\\
    A_7&=7\mathbb{N}-(A_2 \cup A_3 \cup A_5)\\
    \vdots
\end{align*}

You can re-index these sets as $A_1$, $A_2$, $\dots$ if you like.  First we prove that each set has an infinite number of elements. The set $\{p, p^2, p^3, \dots\} \subset A_p$ since no powers of $p$ are in any of the sets $A_q$ for $q < p$.  Clearly $\{p, p^2, p^3, \dots\}$ has infinitely many elements. Next we show that any two sets are pairwise disjoint.  But, this is true by construction: each $A_p$ excluded any integers in a a previous set.



\end{document}